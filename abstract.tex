\begin{flushleft}
\Large
Visual analysis supporting quality improvement of psychological counseling
\end{flushleft}
\begin{flushright}
Tomoya Uetsuji, \\
Graduate School of Engineering, \\
Deparment of Electrical Engineering,\\
Kyoto University
\end{flushright}
\hspace{20zw}
In psychological counseling, treatment is offered to a psychosomatic patient or client mainly due to the physical symptoms caused by stress. However, beginner counselors often have difficulty turning their attention to the cognitive characteristics of the internal problems faced by clients.

 Therefore, expert counselors provide an opportunity for education and training by offering advice to beginner counselors on the content of their counseling conversations. However, due to privacy and client confidentiality issues, experts can rarely obtain voice data from beginner counselors' psychological counseling sessions. Therefore, it is usual for expert counselors to only read direct phonetic transcriptions of conversation of counseling sessions in supervision to support the education of beginner counselors.

However, transcript documents that comprise the text data of conversations in psychological counseling are large in scale and complex. It is very difficult for the supervisor to extract the conversational characteristics from the situation that exists between a beginner counselor and a client by referring to such data. To improve the accessibility of counseling conversations, we propose that visualization of data is more effective. Accordingly, we have developed an appropriate method to visualize the client's answers to each of the beginner counselor's questions. We considered how the limitation in the current supervision for beginner counselors about how to conduct successful counseling conversations can be reduced. In addition, we considered how the supervision for beginner counsellors can be improved using counseling transcripts.

 In addition, if you are interested in changing your own behavior rather than changing the behavior of the opponent, cognitive correction is progressing Therefore, we propose that classifying the client's behavior according to the information provided by the counseling conversation data, and then visualizing the result, reveals the degree of modification of the client's perception and cognition in psychological counseling. we visualized the conversation data using a method that combines a curcular human relationship chart with the behavior enacted by the people discussed in the psychological counseling conversation. We considered that counselors can check the condition of cognitive correction by our proposed time-series visualization of circular-link human relationship chart with behavior arrows.
